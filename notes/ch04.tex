\section*{Data-Parallel Execution Model}
The execution is SPMD, with each thread using unique coordinates
to access some region of the data-structure. A grid is a 3D-array
of blocks and a block is a 3D array of threads.

Each block has a 3-vector \texttt{blockIdx}, and each thread has
a 3-vector \texttt{threadIdx}, the components having identifiers
\texttt{x}, \texttt{y}, \texttt{z}. dimensions of each block and each
grid can be specified using variables of type \texttt{dim3}. If
scalar dimension specifiers are used (e.g. \texttt{int}), then
the \texttt{x} dimension is set to the scalar value while
the \texttt{y} and \texttt{z} are set to \texttt{1}. A grid
can have higher dimensionality then its blocks.

It might be a good practice to assign indices such that the largest
of the dimensions comes first, e.g.
\begin{minted}
{c}
dim3 dimBlock(2, 2, 1);
dim3 dimGrid(4, 2, 2);
KernelFunction<<<dimGrid, dimBlock>>>(...);
\end{minted}

This works better when mapping thread coordinates into data indices
in accessing multidimensional arrays.

NVIDIA GPUs in 2012 could support up to 1024 threads per block with
flexibility in how one can index them.

\subsection*{Mapping Threads To Multidimensional Data}
The programmer needs to think about the dimensionality of the grid
and blocks based on the nature of the data in consideration. E.g.
an image is a 2D grid of values. If the image is greyscale, then
we can do with a representation that takes a scalar value per pixel.

So, if the resolution of our image is $76\times 62$, we could create
a grid using $16\times 16$ blocks, and let the grid be $5\times 4$
blocks. $16\times 16$ is a typical choice of block dimensions for
2D data structures. In this example, we are allocating $80\times 64$
threads to process $76\times 62$ pixels.

To avoid wasteful computations, there should be a conditional guard
so that spawned threads in the grid that do not have pixel values to
work with do not do any computations, like was done in the vector
addition example, wherein four 256-thread blocks were spawned to
deal with a vector of length 1000.

If you have been provided with parameters for a 2D array, e.g. the
dimensions $m\times n$ for the image, then you can use something
like this two spawn a grid for processing the image.
\begin{minted}
{c}
    dim3 dimBlock(ceil(n/16.0), ceil(m/16.0));
    dim3 dimGrid(16, 16, 1);
    pictureKernel<<<dimGrid, dimBlock>>>(d_Pin, d_Pout, n, m);
\end{minted}

So, here's the important bit. We have a global memory pointer
\texttt{d\_Pin} for our input data. As of 2012, CUDA C is based on
ANSI C, which requires the number of rows and columns in 2D arrays
to be known at compile time, which spells doom for most of our
dynamically allocated datastructures. So, the programmer has to
explicitly flatten the tensor to an equivalent 1D array.

The most up to date documents on NVIDIA are on C++. The C documents
are 2010-ish old.The CUDA 11.0 guides on programming and best
practices are exclusive to C++.

Checking C99 compatibility would require some testing. For now, all
I need to know is that C arrays are stored in row-major order.

The following listing shows an example of a CUDA code for matrix
multiplication for square matrices
\begin{code}
\inputminted[samepage=false, breaklines, linenos]{c}{../codes/matrixAdd/matrixMul.cu}
\label{lst:matrixMul}
\caption{Multiplication of square matrices}
\end{code}

\subsection*{Synchronization and Transparent Scalability}
In general one has to coordinate the execution of threads. CUDA allows
threads in the same block to coordinate using a barrier
synchronization function \texttt{\_\_syncthreads()}. All threads
in the calling block will be held until all threads reach the same
location in the program. I see potential deadlocks.

At the very least using this particular barrier synchronization
would allow different blocks to execute independent of each other.

